\documentclass[9pt]{beamer}
\usecolortheme{whale}
\usepackage[polish]{babel}
\usepackage[utf8]{inputenc}
\usepackage[T1]{fontenc}
\usepackage{float}
\usepackage{caption}
\usepackage{graphicx}
\usepackage{enumerate}
\usepackage{wrapfig}

\title{Roboty Zagłady: Czy Nasza Przyszłość jest w Rękach Maszyn?}
\author{Adrian Kwasnik}

\begin{document}

\begin{frame}
\titlepage
\end{frame}




\begin{frame}
\label{r1}
\begin{wrapfigure}{r}{0.35\textwidth}
\includegraphics[scale=0.06]{r1}
AI\footnote{AI-TO SZTUCZNA INTELIGENCJA}
\end{wrapfigure}

Witam serdecznie! Dziś chciałbym poruszyć niezwykle istotny temat, który budzi wiele emocji i obaw – roboty zagłady. Czy nasza przyszłość jest rzeczywiście w rękach maszyn? Przyjrzyjmy się temu z bliska i zastanówmy się, jakie wyzwania stawiają przed nami zaawansowane technologie.
W dzisiejszym świecie, rozwój robotyki osiągnął niebywałe wysokości. Od autonomicznych pojazdów po sztuczną inteligencję, technologia zyskuje na potędze. Jednakże, czy zawsze idziemy w dobrą stronę? Czy istnieje ryzyko, że te zaawansowane maszyny, zamiast służyć ludzkości, mogą stać się narzędziem zagłady?

\end{frame}




\begin{frame}
\label{r2}
\begin{wrapfigure}{r}{0.35\textwidth}
\includegraphics[scale=0.06]{r2}

\end{wrapfigure}

Przyjrzymy się krytycznym zagadnieniom związanym z robotami, które budzą nasze obawy i skłaniają do refleksji. Bądźcie gotowi na podróż w głąb tematu, gdzie odkryjemy zarówno fascynujące, jak i niepokojące aspekty tej nowoczesnej rzeczywistości. Czy jesteśmy gotowi na konfrontację z tym, co przyniesie przyszłość? Zapraszam do wspólnej analizy i dyskusji na temat roli robotów w naszym życiu.Teraz przeniesiemy się głębiej w świat robotów zagłady, aby zrozumieć ich ewolucję od prostych asystentów po potencjalne zagrożenia dla ludzkości.

\end{frame}




\begin{frame}
\label{r3}
\begin{wrapfigure}{r}{0.35\textwidth}
\includegraphics[scale=0.06]{r3}

\end{wrapfigure}

Początkowo roboty miały być naszymi pomocnikami, ułatwiającymi życie codzienne. Jednakże, z biegiem lat, widzimy, że ich zdolności stają się coraz bardziej zaawansowane. Od autonomicznych systemów zarządzania do sztucznej inteligencji zdolnej do samodzielnego uczenia się, roboty zdobywają umiejętności, które jeszcze kilka lat temu wydawały się science fiction.

\end{frame}




\begin{frame}
\label{r4}
\begin{wrapfigure}{r}{0.35\textwidth}
\includegraphics[scale=0.06]{r4}

\end{wrapfigure}

Należy pamiętać, że nasze decyzje dotyczące regulacji i etyki w dziedzinie robotyki będą miały ogromne znaczenie dla kształtowania przyszłości. Czy możemy znaleźć równowagę między korzyściami, jakie niesie zaawansowana technologia, a potencjalnymi zagrożeniami dla ludzkości? Zapraszam do refleksji nad tymi kwestiami podczas naszej podróży przez fascynujący, lecz jednocześnie trudny świat robotów zagłady.

\end{frame}




\begin{frame}

\begin{figure}
\frametitle{Czy możemy kontrolować rozwój sztucznej inteligencji? Jakie wyzwania stawia przed nami coraz większa autonomiczność maszyn?}
\includegraphics[scale=0.08]{r1}
\end{figure}

\end{frame}



\begin{frame}
\frametitle{Poziom zagrozenia !}
\begin{table}
    \begin{tabular}{|c|c|l|}
      \hline
      Mozliwosci & Poziom & Wydarzenie \\
      \hline
      Wylaczenie AI & 1 & AI dokonuje sie karalnych czynow \\
      Przeciwdzialanie zbrojne AI & 2 & AI dokonuje zniszczen \\
      brak & 3 & AI wyrwalo sie spod kontroli\\
      \hline
    \end{tabular}
    \caption{Tabela zagrozenia AI}
  \end{table}
\end{frame}

\begin{frame}
\frametitle{Plusy AI\cite{Plusy}}
\begin{enumerate}[a)]
\item Jest przydatne dla naukowcow
\pause
\item  Jest przydatne dla tworcow/artystow
\pause
\item Rozrywka ogolnie
\pause
\item Automatyzuje swiat
\pause
\end{enumerate}

\end{frame}

\begin{frame}
\frametitle{Minusy AI\cite{Minusy}}
\begin{enumerate}[1]
\item Rosnąca złożoność systemów AI niesie ze sobą wyzwania związane z bezpieczeństwem i prywatnością danych. Może to obejmować ryzyko ataków hakerskich, wycieków danych oraz wykorzystywania informacji w niepożądany sposób.
\item  Rosnąca złożoność systemów AI niesie ze sobą wyzwania związane z bezpieczeństwem i prywatnością danych. Może to obejmować ryzyko ataków hakerskich, wycieków danych oraz wykorzystywania informacji w niepożądany sposób.
\item Rosnąca złożoność systemów AI niesie ze sobą wyzwania związane z bezpieczeństwem i prywatnością danych. Może to obejmować ryzyko ataków hakerskich, wycieków danych oraz wykorzystywania informacji w niepożądany sposób.
\item Rosnąca złożoność systemów AI niesie ze sobą wyzwania związane z bezpieczeństwem i prywatnością danych. Może to obejmować ryzyko ataków hakerskich, wycieków danych oraz wykorzystywania informacji w niepożądany sposób.
\end{enumerate}
\end{frame}

\begin{frame}
\frametitle{ODWOLANIA I BIBLIOGRAFIA}
\begin{enumerate}[a]
\item Zdjecie ze slajdu \ref{r1}
\item Zdjecie ze slajdu \ref{r2}
\item Zdjecie ze slajdu \ref{r3}
\item Zdjecie ze slajdu \ref{r4}
\end{enumerate}
\begin{thebibliography}{99}
\bibitem{Plusy}
Wiadomo wszystko ma plusy
\bibitem{Minusy}
Wiadomo wszystko ma minusy
\end{thebibliography}
\end{frame}


\end{document}
